%%%%%%%%%%%%%%%%%%%%%%%%%%%%%%%%%%%%%%%%%
% Wenneker Article
% LaTeX Template
% Version 2.0 (28/2/17)
%
% This template was downloaded from:
% http://www.LaTeXTemplates.com
%
% Authors:
% Vel (vel@LaTeXTemplates.com)
% Frits Wenneker
%
% License:
% CC BY-NC-SA 3.0 (http://creativecommons.org/licenses/by-nc-sa/3.0/)
%
%%%%%%%%%%%%%%%%%%%%%%%%%%%%%%%%%%%%%%%%%

\newcommand{\py}[1]{{\tt #1}}

%----------------------------------------------------------------------------------------
%	PACKAGES AND OTHER DOCUMENT CONFIGURATIONS
%----------------------------------------------------------------------------------------

\documentclass[10pt, a4paper, twocolumn]{article} % 10pt font size (11 and 12 also possible), A4 paper (letterpaper for US letter) and two column layout (remove for one column)

\input{structure.tex} % Specifies the document structure and loads requires packages

%----------------------------------------------------------------------------------------
%	ARTICLE INFORMATION
%----------------------------------------------------------------------------------------

\title{PyVallex: A Processing System for Vallency Lexicon Data} % The article title

\author{
	\authorstyle{Anna Vernerová\textsuperscript{1} and Jonathan Verner\textsuperscript{2,3}} % Authors
	\newline\newline % Space before institutions
	\textsuperscript{1}\institution{Institute of Formal and Applied Linguistics, Charles University, Prague}\\ % Institution 1
	\textsuperscript{2}\institution{Department of Logic, Charles University, Prague}\\ % Institution 2
	\textsuperscript{3}\institution{Institute of Formal and Applied Linguistics, Charles University, Prague} % Institution 3
}

% Example of a one line author/institution relationship
%\author{\newauthor{John Marston} \newinstitution{Universidad Nacional Autónoma de México, Mexico City, Mexico}}

\date{\today} % Add a date here if you would like one to appear underneath the title block, use \today for the current date, leave empty for no date

%----------------------------------------------------------------------------------------

\begin{document}

\maketitle % Print the title

\thispagestyle{firstpage} % Apply the page style for the first page (no headers and footers)

%----------------------------------------------------------------------------------------
%	ABSTRACT
%----------------------------------------------------------------------------------------

\lettrineabstract{We present PyVallex a Python-based system for display, editing and processing
of vallency lexicon data. The system consists of several components:
a parser for the specialized lexicon format used in several vallency lexicons,
a data-validation framework, a regular expression based search engine, 
a map-reduce style framework for querying the lexicon data and a web-based interface integrating
complex search and some basic editing capabilities. Several of the components are provided
as a Python package allowing for programmatic access to the data.
}

%----------------------------------------------------------------------------------------
%	PREVIOUS WORK
%----------------------------------------------------------------------------------------
\section{Previous work}

%----------------------------------------------------------------------------------------
%	DESIGN OF THE SYSTEM
%----------------------------------------------------------------------------------------
\section{Motivation \& Design}
The main motivation for designing a system from scratch was that the mostly unstructured 
nature of the previous system made it difficult to add new functionality and maintain the
system. The design of the system did not allow for easy integration into other tools which
was compounded by almost non-existing documentation. On the other hand, the older system
provided significant functionality which the new system needed to replicate.

The above considerations lead to the following design goals for the new system:

\paragraph{Modularity} To make adding new functionality easier, the system should be structured
into components, each with a well defined public interface (API). The components should only communicate
with each other through this API. This allows any new functionality to modify only the
relevant parts of the system and thus lowers the barrier for new contributors---a developer does
not need to have precise knowledge of the whole system to add/modify functionality to a single component.

\paragraph{Extendability} While the system should provide as much as possible of the functionality
already present in the older system, it should be designed from the start to be flexible and easily 
extendible. In particular, it should have a well-defined data-model and provide an interface to allow
accessing the data from other programs (either through a library interface or, e.g., a REST API).

\paragraph{Maintainability} The design should aim to minimize the maintenance cost of the system.
In particular the system should be well documented, the components should be covered by an automated
test suite and the code style should be as uniform as possible and follow best practices.

Given the above goals, we have decided to implement the system in the \emph{Python 3} programming 
language. Its advantage over \emph{Perl} (the language of the previous system) is that it provides
much better language support for structured programming. It is also becoming much more popular in
the NLP community, which makes it more likely that new contributors will be able to work with the
code. Although Python (as is the case also for Perl) is a weakly typed language, we have opted
to enforce the use of type hints via a static type checker (\cite{tool:mypy}) run before every commit. 
This makes the code effectively strongly-typed. In the interest of maintainability, we also enforce 
a uniform coding style through the use of a commit-hook which runs autopep (\cite{tool:autopep8}) 
on the commited code. 



%----------------------------------------------------------------------------------------
%	THE DATA LAYER
%----------------------------------------------------------------------------------------
\section{Data Layer}
A vallency lexicon consists of a collection of \emph{Lexemes}. A Lexeme represents a group of related
lexical units, each unit corresponding to a single word-meaning pair. Each lexical unit can be annotated
with various linguistic properties.

The data layer definition is provided in the \py{vallex.data\_structures} module and consists of the
following classes \py{Lexicon} (representing a collection of \emph{Lexemes}), \py{Lexeme} (representing
a single Lexeme), \py{LexicalUnit} (representing a lexical unit), \py{Attrib} (representing a linguistically
relevant property of a lexical unit). The \py{Attrib} class also has several specializations: \py{Frame}
(representing the frame annotation), \py{Lemma} (representing the lemma annotation) and \py{SpecVal}
(representing the changes between a verb-noun derived pair). It is expected, that further sepcializations
will be defined to deal with particular properties.

Each of the above classes can store annotator provided textual comments (which can be used to explain
the reasoning behind a specific annotation, mark the data element as a work in progress, etc). They can also
store the original unparsed form (in the specialized annotator format) together with its location in the 
source and they all provide a method to convert the data into a JSON representable structure.

%----------------------------------------------------------------------------------------
%	CORE COMPONENTS
%----------------------------------------------------------------------------------------
\section{Core components}
The core of the system consists of a \emph{parser module} which takes care of parsing the data and
constructing an in-memory representation of it---the data layer; a \emph{search module} which
provides query capabilities; a \emph{script module} which provides a framework for running data validation
and batch processing scripts; a \emph{output module} which converts the in-memory representation into
various output formats

%------------------------------------------------
\subsection{The parser}
%------------------------------------------------
\subsection{Search}
%------------------------------------------------
\subsection{Data Validation}
%------------------------------------------------
\subsection{Batch processing}
%------------------------------------------------
\subsection{Output}

%----------------------------------------------------------------------------------------
%	THE UI LAYER
%----------------------------------------------------------------------------------------

\section{The UI Layer}

%----------------------------------------------------------------------------------------
%	FUTURE WORK
%----------------------------------------------------------------------------------------


\section{Future work}


This sentence requires citation \citep{Reference1}. This sentence requires multiple citations to imply that it is better supported \citep{Reference2,Reference3}. Finally, when conducting an appeal to authority, it can be useful to cite a reference in-text, much like \cite{Reference1} do quite a bit. Oh, and make sure to check out the bear in Figure \ref{bear}.

Lorem ipsum dolor sit amet, consectetur adipiscing elit. Fusce maximus nisi ligula. Morbi laoreet ex ligula, vitae lobortis purus mattis vel. Vestibulum ante ipsum primis in faucibus orci luctus et ultrices posuere cubilia Curae; Donec ac metus ut turpis mollis placerat et nec enim. Duis tristique nibh maximus faucibus facilisis. Praesent in consequat leo. Maecenas condimentum ex rhoncus, elementum diam vel, malesuada ante. Fusce pulvinar, mauris pretium placerat venenatis, lectus ex tempus lacus, id suscipit libero lorem eu augue. Interdum et malesuada fames ac ante ipsum primis in faucibus.

Aliquam elementum nulla at arcu finibus aliquet. Praesent congue ultrices nisl pretium posuere. Nunc vel nulla hendrerit, ultrices justo ut, ultrices sapien. Duis ut arcu at nunc pellentesque consectetur. Vestibulum eget nisl porta, ultricies orci eget, efficitur tellus. Maecenas rhoncus purus vel mauris tincidunt, et euismod nibh viverra. Mauris ultrices tellus quis ante lobortis gravida. Duis vulputate viverra erat, eu sollicitudin dui. Proin a iaculis massa. Nam at turpis in sem malesuada rhoncus. Aenean tempor risus dui, et ultrices nulla rutrum ut. Nam commodo fermentum purus, eget mattis odio fringilla at. Etiam congue et ipsum sed feugiat. Morbi euismod ut purus et tempus. Etiam est ligula, aliquam eget porttitor ut, auctor in risus. Curabitur at urna id dui lobortis pellentesque.

\begin{align}
	A = 
	\begin{bmatrix}
		A_{11} & A_{21} \\
		A_{21} & A_{22}
	\end{bmatrix}
\end{align}

Donec nec nibh sagittis, finibus mauris quis, laoreet augue. Maecenas aliquam sem nunc, vel semper urna hendrerit nec. Pellentesque habitant morbi tristique senectus et netus et malesuada fames ac turpis egestas. Maecenas pellentesque dolor lacus, sit amet pretium felis vestibulum finibus. Duis tincidunt sapien faucibus nisi vehicula tincidunt. Donec euismod suscipit ligula a tempor. Aenean a nulla sit amet magna ullamcorper condimentum. Fusce eu velit vitae libero varius condimentum at sed dui.

In non leo tincidunt, tristique orci eu, suscipit ex. Fusce non lectus ut dolor tincidunt fermentum. Donec dictum mauris magna, ut dictum nisl finibus quis. Nulla elementum ipsum ut lectus sodales finibus. Nulla ac malesuada magna. Etiam arcu dolor, luctus eget elit a, volutpat vulputate mi. Donec elementum tellus libero, ut ornare orci dignissim lacinia. Nullam iaculis vehicula sem, at tempor tellus. Praesent eu nisi a elit viverra lobortis. Nullam eu metus et justo molestie posuere vitae imperdiet erat. Praesent at gravida dui. Vivamus mauris odio, efficitur eget lacus quis, mattis tristique risus. Mauris quis metus sed risus lobortis sollicitudin vitae vitae quam. Morbi leo turpis, aliquam at nunc sit amet, ultricies dictum lorem. Nam et fringilla elit. Vestibulum auctor, turpis ut facilisis tempor, arcu nibh tincidunt libero, quis blandit leo turpis a urna.

%------------------------------------------------

\subsection{Subsection}

Nam ante risus, tempor nec lacus ac, congue pretium dui. Donec a nisl est. Integer accumsan mauris eu ex venenatis mollis. Aliquam sit amet ipsum laoreet, mollis sem sit amet, pellentesque quam. Aenean auctor diam eget erat venenatis laoreet. In ipsum felis, tristique eu efficitur at, maximus ac urna. Aenean pulvinar eu lorem eget suscipit. Aliquam et lorem erat. Nam fringilla ante risus, eget convallis nunc pellentesque non. Donec ipsum nisl, consectetur in magna eu, hendrerit pulvinar orci. Mauris porta convallis neque, non viverra urna pulvinar ac. Cras non condimentum lectus. Aliquam odio leo, aliquet vitae tellus nec, imperdiet lacinia turpis. Nam ac lectus imperdiet, luctus nibh a, feugiat urna.

\begin{itemize}
	\item First item in a list 
	\item Second item in a list 
	\item Third item in a list
\end{itemize}

Nunc egestas quis leo sed efficitur. Donec placerat, dui vel bibendum bibendum, tortor ligula auctor elit, aliquet pulvinar leo ante nec tellus. Praesent at vulputate libero, sit amet elementum magna. Pellentesque sodales odio eu ex interdum molestie. Suspendisse lacinia, augue quis interdum posuere, dolor ipsum euismod turpis, sed viverra nibh velit eget dolor. Curabitur consectetur tempus lacus, sit amet luctus mauris interdum vel. Curabitur vehicula convallis felis, eget mattis justo rhoncus eget. Pellentesque et semper lectus.

\begin{description}
	\item[First] This is the first item
	\item[Last] This is the last item
\end{description}

Donec nec nibh sagittis, finibus mauris quis, laoreet augue. Maecenas aliquam sem nunc, vel semper urna hendrerit nec. Pellentesque habitant morbi tristique senectus et netus et malesuada fames ac turpis egestas. Maecenas pellentesque dolor lacus, sit amet pretium felis vestibulum finibus. Duis tincidunt sapien faucibus nisi vehicula tincidunt. Donec euismod suscipit ligula a tempor. Aenean a nulla sit amet magna ullamcorper condimentum. Fusce eu velit vitae libero varius condimentum at sed dui.

%------------------------------------------------

\subsection{Subsection}

In hac habitasse platea dictumst. Etiam ac tortor fermentum, ultrices libero gravida, blandit metus. Vivamus sed convallis felis. Cras vel tortor sollicitudin, vestibulum nisi at, pretium justo. Curabitur placerat elit nunc, sed luctus ipsum auctor a. Nulla feugiat quam venenatis nulla imperdiet vulputate non faucibus lorem. Curabitur mollis diam non leo ullamcorper lacinia.

Morbi iaculis posuere arcu, ut scelerisque sem. Class aptent taciti sociosqu ad litora torquent per conubia nostra, per inceptos himenaeos. Mauris placerat urna id enim aliquet, non consequat leo imperdiet. Phasellus at nibh ut tortor hendrerit accumsan. Phasellus sollicitudin luctus sapien, feugiat facilisis risus consectetur eleifend. In quis luctus turpis. Nulla sed tellus libero. Pellentesque metus tortor, convallis at tellus quis, accumsan faucibus nulla. Fusce auctor eleifend volutpat. Maecenas vel faucibus enim. Donec venenatis congue congue. Integer sit amet quam ac est aliquam aliquet. Ut commodo justo sit amet convallis scelerisque.

\begin{enumerate}
	\item First numbered item in a list
	\item Second numbered item in a list
	\item Third numbered item in a list
\end{enumerate}

Aliquam elementum nulla at arcu finibus aliquet. Praesent congue ultrices nisl pretium posuere. Nunc vel nulla hendrerit, ultrices justo ut, ultrices sapien. Duis ut arcu at nunc pellentesque consectetur. Vestibulum eget nisl porta, ultricies orci eget, efficitur tellus. Maecenas rhoncus purus vel mauris tincidunt, et euismod nibh viverra. Mauris ultrices tellus quis ante lobortis gravida. Duis vulputate viverra erat, eu sollicitudin dui. Proin a iaculis massa. Nam at turpis in sem malesuada rhoncus. Aenean tempor risus dui, et ultrices nulla rutrum ut. Nam commodo fermentum purus, eget mattis odio fringilla at. Etiam congue et ipsum sed feugiat. Morbi euismod ut purus et tempus. Etiam est ligula, aliquam eget porttitor ut, auctor in risus. Curabitur at urna id dui lobortis pellentesque.

\begin{table}
	\caption{Example table}
	\centering
	\begin{tabular}{llr}
		\toprule
		\multicolumn{2}{c}{Name} \\
		\cmidrule(r){1-2}
		First Name & Last Name & Grade \\
		\midrule
		John & Doe & $7.5$ \\
		Richard & Miles & $5$ \\
		\bottomrule
	\end{tabular}
\end{table}

%------------------------------------------------

\section{Section}

\begin{figure}
	\includegraphics[width=\linewidth]{bear.jpg} % Figure image
	\caption{A majestic grizzly bear} % Figure caption
	\label{bear} % Label for referencing with \ref{bear}
\end{figure}

In hac habitasse platea dictumst. Vivamus eu finibus leo. Donec malesuada dui non sagittis auctor. Aenean condimentum eros metus. Nunc tempus id velit ut tempus. Quisque fermentum, nisl sit amet consectetur ornare, nunc leo luctus leo, vitae mattis odio augue id libero. Mauris quis lectus at ante scelerisque sollicitudin in eu nisi. Nulla elit lacus, ultricies eu erat congue, venenatis semper turpis. Ut nec venenatis velit. Mauris lacinia diam diam, ac egestas neque sodales sed. Curabitur eu diam nulla. Duis nec turpis finibus, commodo diam sed, bibendum erat. Nunc in velit ullamcorper, posuere libero a, mollis mauris. Nulla vehicula quam id tortor ornare blandit. Aenean maximus tempor orci ultrices placerat. Aenean condimentum magna vulputate erat mattis feugiat.

Quisque lacinia, purus id mattis gravida, sem enim fringilla erat, non dapibus est tellus pellentesque velit. Vivamus pretium sem quis leo placerat, at dignissim ex iaculis. Donec neque tortor, pharetra quis vestibulum id, tempus scelerisque mi. Cras in mattis est. Integer nec lorem rutrum, semper ligula bibendum, iaculis neque. Sed in nunc placerat, viverra dui in, fringilla sem. Sed quis rutrum magna, vitae pellentesque eros.

Praesent maximus mauris vitae nisl pulvinar, at tristique tortor aliquam. Etiam sit amet nunc in nulla vulputate sollicitudin. Aliquam erat volutpat. Praesent pharetra gravida cursus. Quisque vulputate lacus nunc. Integer orci ex, porttitor quis sapien id, eleifend gravida mi. Etiam efficitur justo eget nulla congue mattis. Duis commodo vel arcu a pretium. Aenean eleifend viverra nisl, nec ornare lacus rutrum in.

Vivamus pulvinar ac eros eu pellentesque. Duis nibh felis, sagittis sed lacus at, sagittis mattis nisi. Fusce ante dui, tincidunt in scelerisque ut, sagittis at magna. Fusce tincidunt felis et odio tincidunt imperdiet. Cras ut facilisis nisl. Aliquam vitae consequat metus, eget gravida augue. In imperdiet justo quis nulla venenatis accumsan. Aliquam aliquet consectetur tortor, at sollicitudin sapien porta sed. Donec efficitur mauris id rhoncus volutpat. Vestibulum ante ipsum primis in faucibus orci luctus et ultrices posuere cubilia Curae; Sed bibendum purus dapibus tincidunt euismod. Nullam malesuada ultrices lacus, ut tincidunt dolor. Etiam imperdiet quam eget elit tincidunt scelerisque. Curabitur ut ullamcorper dui. Cras gravida porta leo, ut lobortis nisl venenatis pulvinar. Proin non semper nulla.

Praesent pretium nisl purus, id mollis nibh efficitur sed. Sed sit amet urna leo. Nulla sed imperdiet sem. Donec ut diam tristique, faucibus ligula vel, varius est. In ipsum ligula, elementum vitae velit ac, viverra tincidunt enim. Phasellus gravida diam id nisl interdum maximus. Ut semper, tortor vitae congue pharetra, justo odio commodo urna, vel tempus libero ex et risus. Vivamus commodo felis non venenatis rutrum. Sed pulvinar scelerisque augue in porta. Sed maximus libero nec tellus malesuada elementum. Proin non augue posuere, pellentesque felis viverra, varius urna. Lorem ipsum dolor sit amet, consectetur adipiscing elit. Donec dignissim urna eget diam dictum, eget facilisis libero pulvinar.

Aliquam ex tellus, hendrerit sed odio sit amet, facilisis elementum enim. Suspendisse potenti. Integer molestie ac augue sit amet fermentum. Vivamus ultrices ante nulla, vitae venenatis ipsum ullamcorper sed. Phasellus gravida felis sapien, ac porta purus pharetra quis. Sed eget augue tellus. Nam vitae hendrerit arcu, id iaculis ipsum. Pellentesque sed magna tortor.

In ac tempus diam. Sed nec lobortis massa, suscipit accumsan justo. Quisque porttitor, ligula a semper euismod, urna diam dictum sem, sed maximus risus purus sit amet felis. Fusce elementum maximus nisi a mattis. Nulla vitae elit erat. Integer sit amet commodo risus, eget elementum nulla. Donec ultricies erat sit amet sem commodo iaculis. Donec euismod volutpat lacus, ut tempor est lacinia a. Vivamus auctor condimentum tincidunt. Praesent sed finibus urna. Sed pellentesque blandit magna et rhoncus.

Integer vel turpis nec tellus sodales malesuada a vel odio. Fusce et lectus eu nibh rhoncus tempus vel nec elit. Suspendisse commodo orci velit, lacinia dictum odio accumsan et. Vivamus libero dui, elementum vel nibh non, fermentum venenatis risus. Aliquam sed sapien ac orci sodales tempus a eget dui. Morbi non dictum tortor, quis tincidunt nibh. Proin ut tincidunt odio.

Pellentesque ac nisi dolor. Pellentesque maximus est arcu, eu scelerisque est rutrum vitae. Mauris ullamcorper vulputate vehicula. Praesent fermentum leo ac velit accumsan consectetur. Aliquam eleifend ex eros, ut lacinia tellus volutpat non. Pellentesque sit amet cursus diam. Maecenas elementum mattis est, in tincidunt ex pretium ac. Integer ultrices nunc rutrum, pretium sapien vitae, lobortis velit.

%----------------------------------------------------------------------------------------
%	BIBLIOGRAPHY
%----------------------------------------------------------------------------------------

\printbibliography[title={Bibliography}] % Print the bibliography, section title in curly brackets

%----------------------------------------------------------------------------------------

\end{document}
